\documentclass[titlepage]{article}
\usepackage{babel}
\usepackage{amsmath}
\usepackage{amssymb}
\usepackage{amsthm}
\usepackage{multicol} %spalten in seite
\usepackage{graphicx} %bilder einfügen
\usepackage{tabto} %tabulator mit \tab
\usepackage{hyperref}
\usepackage[T1]{fontenc}
\usepackage{mathrsfs}  
\usepackage[utf8]{inputenc}
\usepackage{listings} %quellcode
\pagestyle{plain}
\pagenumbering{arabic}
\renewcommand{\arraystretch}{1.3} %vertikaler abstand von tabellen
\newcommand{\n}{\newline}
\usepackage[left=20mm, right=15mm, top=25mm, bottom=30mm, paper=a4paper]{geometry}
\renewcommand{\n}{\newline}
\newcommand{\A}{\mathbb{A}}
\newcommand{\B}{\mathbb{B}}
\newcommand{\C}{\mathbb{C}}
\newcommand{\D}{\mathbb{D}}
\newcommand{\E}{\mathbb{E}}
\newcommand{\F}{\mathbb{F}}
\newcommand{\G}{\mathbb{G}}
\renewcommand{\H}{\mathbb{H}}
\newcommand{\I}{\mathbb{I}}
\newcommand{\J}{\mathbb{J}}
\newcommand{\K}{\mathbb{K}}
\renewcommand{\L}{\mathbb{L}}
\newcommand{\M}{\mathbb{M}}
\newcommand{\N}{\mathbb{N}}
\renewcommand{\O}{\mathbb{O}}
\renewcommand{\P}{\mathbb{P}}
\newcommand{\Q}{\mathbb{Q}}
\newcommand{\R}{\mathbb{R}}
\renewcommand{\S}{\mathbb{S}}
\newcommand{\T}{\mathbb{T}}
\newcommand{\U}{\mathbb{U}}
\newcommand{\V}{\mathbb{V}}
\newcommand{\W}{\mathbb{W}}
\newcommand{\X}{\mathbb{X}}
\newcommand{\Y}{\mathbb{Y}}
\newcommand{\Z}{\mathbb{Z}}

\begin{document}
	
	\title{Lineare Algebra - Übung 06}
	\author{Felix Tischler, Martrikelnummer: 191498}
	\date{\today}
	\maketitle
	
	\section*{6.1: Lineares Gleichungssystem}
	$A$ und $\vec{b}$ werden zunächst in die erweitere Matrixform gebracht und dann in ZSF:
		\begin{align*}
			\left(\begin{array}{cccccc|c}
				1&1&1&1&1&1&0\\
				0&-3&4&1&1&2&-1\\
				1&1&1&2&2&2&0\\
				-5&-11&3&-6&-6&-4&-2
			\end{array}\right)
			\overset{(1)}{\rightsquigarrow}
			\left(\begin{array}{cccccc|c}
				1&1&1&1&1&1&0\\
				0&-3&4&1&1&2&-1\\
				0&0&0&1&1&1&0\\
				0&-6&8&-1&-1&1&-2
			\end{array}\right)\\\\
			\overset{(2)}{\rightsquigarrow}
			\left(\begin{array}{cccccc|c}
				1&1&1&1&1&1&0\\
				0&-3&4&1&1&2&-1\\
				0&0&0&1&1&1&0\\
				0&0&0&-3&-3&-3&0
			\end{array}\right)
			\overset{(3)}{\rightsquigarrow}
			\left(\begin{array}{cccccc|c}
				1&1&1&1&1&1&0\\
				0&-3&4&1&1&2&-1\\
				0&0&0&1&1&1&0\\
				0&0&0&0&0&0&0
			\end{array}\right)
		\end{align*}
	(1) $III - I \text{ und } IV + 5\cdot I$ \\
	(2) $IV - 2\cdot II$\\
	(3) $IV + 3\cdot III$\\\\
	Somit sind $x_1,x_2,x_4$ gebundene Variablen und $x_3,x_5.x_6$ freie Variablen. Die vierte Zeile kann ignoriert werden, da sie kein Pivot Element enthält.
	\begin{align*}
		\left(\begin{array}{cccccc|c}
			1&1&1&1&1&1&0\\
			0&-3&4&1&1&2&-1\\
			0&0&0&1&1&1&0
		\end{array}\right)
		\Rightarrow
		\begin{array}{cccccccccccccc}
			I&x_1&+&x_2&+&x_3&+&x_4&+&x_5&+&x_6&=&0\\
			I&-3x_2&+&4x_3&+&x_4&+&x_5&+&2x_6&=&-1\\
			III&x_4&+&x_5&+&x_6&=&0
		\end{array}
	\end{align*}
	Aus III folgt: $x_4=-x_5-x_6$, aus II folgt: $x_2=\frac{1}{3}+\frac{4}{3}x_3+\frac{1}{3}x_6$, aus I folgt: $x_1=-\frac{1}{3}-\frac{7}{3}x_3-\frac{1}{3}x_6$.\\\\
	D.h.:
	\begin{align*}
		LR(A;\vec{b})=\{\begin{pmatrix}
			-1/3\\1/3\\0\\0\\0\\0
		\end{pmatrix}+\begin{pmatrix}
		-7/3&0&-1/3\\
		4/3&0&1/3\\
		1&0&0\\
		0&-1&-1\\
		0&1&0\\
		0&0&1
	\end{pmatrix}\cdot\vec{c}\mid\vec{c}\in\R^3\}
	\end{align*}
	\section*{6.2: Gleichungssysteme mit Parametern}
	$A$ und $\vec{b}$ werden zunächst in die erweitere Matrixform gebracht und dann in ZSF:
		\begin{align*}
			\left(\begin{array}{ccc|c}
				1&1&0&1\\
				-3&-1&-2&5\\
				-3&s&-1&t
			\end{array}\right)
			\overset{(1)}{\rightsquigarrow}
			\left(\begin{array}{ccc|c}
				1&1&0&1\\
				0&2&-2&8\\
				0&s+3&-1&t+3
			\end{array}\right)
			\overset{(2)}{\rightsquigarrow}
			\left(\begin{array}{ccc|c}
				1&1&0&1\\
				0&1&-1&4\\
				0&s+3&-1&t+3
			\end{array}\right)\\\\
			\overset{(3)}{\rightsquigarrow}
			\left(\begin{array}{ccc|c}
				1&1&0&1\\
				0&1&-1&4\\
				0&0&s+2&t-4s-9
			\end{array}\right)
			\Rightarrow
			\begin{array}{cccccc}
				I&x_1&+&x_2&=&1\\
				II&x_2&-&x_3&=&4\\
				III&&&(s+2)x_3&=&t-4s-9
			\end{array}
		\end{align*}
	Fall eins: $s=-2$ und $t-4s-9\neq0$:
	\begin{align*}
		LR(A;\vec{b})=\varnothing
	\end{align*}
	Fall zwei: $s=-2$ und $t-4s-9=0$:
	\begin{align*}
		&\left(\begin{array}{ccc|c}
			1&1&0&1\\
			0&1&-1&4\\
			0&0&0&0
		\end{array}\right)
		\Rightarrow
		\begin{matrix}
			I&x_1&+&x_2&=&1\\
			II&x_2&-&x_3&=&4
		\end{matrix}
	\end{align*}
	aus II folgt $x_2=4+x_3$ und aus I folgt mit $x_2$ eingesetzt $x_1=-3-x_3$.\\\\ D.h.:
	\begin{align*}
		LR(A;\vec{b})=\{\begin{pmatrix}
			-3\\4\\0
		\end{pmatrix}+x_3\begin{pmatrix}
		-1\\1\\1
	\end{pmatrix}\mid x_3\in\R\}&&
	\end{align*}\qed\\
	Fall drei: $s\neq-2$:
	\begin{align*}
		&\left(\begin{array}{ccc|c}
			1&1&0&1\\
			0&1&-1&4\\
			0&0&s+2&t-4s-9
		\end{array}\right)
		\Rightarrow
		\begin{matrix}
			I&x_1&+&x_2&=&1\\
			II&x_2&-&x_3&=&4\\
			III&&&(s+2)x_3&=&t-4s-9
		\end{matrix}
	\end{align*}
	Aus III folgt $x_3=\frac{t-4s-9}{s+2}$, aus II folgt $x_2=\frac{t-1}{s+2}$, aus I folgt $x_1=\frac{s-t+3}{s+2}$\\\\
	D.h.:
	\begin{align*}
		LR(A;\vec{b})=\{\begin{pmatrix}
		\frac{2-t+3}{s+2}\\\frac{t-1}{s+2}\\\frac{t-4s-9}{s+2}
	\end{pmatrix}\mid s,t\in\R\}&&	
	\end{align*}\qed
	\section*{6.3: Linearkombinationen}
		\subsection*{a)}
			\begin{align*}
				\begin{pmatrix}
					0&1&2\\
					1&1&1\\
					2&2&0
				\end{pmatrix}
				\overset{(1)}{\rightsquigarrow}
				\begin{pmatrix}
					0&1&2\\
					1&1&1\\
					1&1&0
				\end{pmatrix}
				\overset{(2)}{\rightsquigarrow}
				\begin{pmatrix}
					0&1&2\\
					0&0&1\\
					1&1&0
				\end{pmatrix}
				\overset{(3)}{\rightsquigarrow}
				\begin{pmatrix}
					1&1&0\\
					0&1&2\\
					0&0&1
				\end{pmatrix}
			\end{align*}
		(1) III $\cdot$$\frac{1}{2}$\\
		(2) II - III\\
		(3) III mit I tauschen und II mit neuer III tauschen\\\\
		Es sind alle Zeilen der Matrix Pivotzeilen, also sind die Vektoren lin. unabhängig.\\\\ Nachweis:
		\begin{align*}
			I& &x_1+x_2&=0&&&&&&&&&\\
			II& &x_2+2x_3&=0\\
			III& &x_3&=0
		\end{align*}
		Aus III folgt $x_3=0$ dies in II eingesetzt ergibt $x_2=0$. $x_2,x_3$ in I eingesetzt ergibt $x_1=0$ \qed
		\subsection*{b)}
			\begin{align*}
				\begin{pmatrix}
					1&5&1\\
					2&4&1\\
					1&-1&0
				\end{pmatrix}
				\overset{(1)}{\rightsquigarrow}
				\begin{pmatrix}
					1&5&1\\
					0&6&1\\
					1&-1&0
				\end{pmatrix}
				\overset{(2)}{\rightsquigarrow}
				\begin{pmatrix}
					1&5&1\\
					0&6&1\\
					0&-6&-1
				\end{pmatrix}
				\overset{(3)}{\rightsquigarrow}
				\begin{pmatrix}
					1&5&1\\
					0&6&1\\
					0&0&0
				\end{pmatrix}
			\end{align*}
		Die drei Vektoren sind also lin. Abhängig, da es nur 2 Pivotzeilen gibt. Sie spannen eine Ebene im Raum auf. Jener Vektor welcher nicht in der Ebene liegt kann nicht als lin. Kombination dargestellt werden. Ich berechne den Normalenvektor, da er senkrecht zur Ebene steht und somit nicht in der Ebene liegt.:
		\begin{align*}
			\begin{pmatrix}
				1\\2\\1
			\end{pmatrix}\times
			\begin{pmatrix}
				1\\1\\0
			\end{pmatrix}=
			\begin{pmatrix}
				2\cdot0-1\cdot1\\1\cdot0-1\cdot1\\1\cdot1-1\cdot2
			\end{pmatrix}=
			\begin{pmatrix}
				-1\\1\\-1
			\end{pmatrix}
		\end{align*}\qed\\
		Dies ist ein konkretes Beispiel. Es gibt unendlich viele.
\end{document}
