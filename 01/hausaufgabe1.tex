\documentclass[titlepage]{article}
\usepackage{babel}
\usepackage{amsmath}
\usepackage{amssymb}
\usepackage{amsthm}
\usepackage{hyperref}
\pagestyle{plain}
\pagenumbering{arabic}
\renewcommand{\arraystretch}{1.3} %vertikaler abstand von tabellen
\newcommand{\n}{\newline}

\begin{document}
	
	\title{Hausaufgaben Lineare Algebra 01}
	\author{Felix Tischler, Martrikelnummer: 191498}
	\date{\today}
	\maketitle
	
	\section*{1.1.a}
		{
			Es sein P und Q Aussagen. \\ \\
			\begin{tabular}{cccc|cc|cc}
				$P$&$Q$&$\neg P$&$\neg Q$&$\neg (P\wedge Q)$&$\neg (P\vee Q)$&$(\neg P)\vee (\neg Q)$&$(\neg P)\wedge (\neg Q)$
				\\\hline
				$F$&$F$&$W$&$W$&$W$&$W$&$W$&$W$
				\\
				$F$&$W$&$W$&$F$&$W$&$F$&$W$&$F$		
				\\
				$W$&$F$&$F$&$W$&$W$&$F$&$W$&$F$
				\\
				$W$&$W$&$F$&$F$&$F$&$F$&$F$&$F$
			\end{tabular}
		}
		\n \n \n
		\hspace*{10mm} d.h.:
		\n \n
		\hspace*{20mm}
		$\neg (P\wedge Q)\Leftrightarrow (\neg P)\vee (\neg Q)$
		\n \n \hspace*{20mm}
		$\neg (P\vee Q)\Leftrightarrow (\neg P)\wedge (\neg Q)$
		
		\section*{1.1.b}
		{
			\centering
			\begin{tabular}{ccccc|cc}
				$P$&$Q$&$\neg P$&$\neg P\wedge Q$&$Q\Rightarrow P$&$(P\wedge (Q\Rightarrow P))$&$((P\wedge Q)\vee (\neg P))$
				\\\hline
				$F$&$F$&$W$&$F$&$W$&$F$&$W$
				\\
				$F$&$W$&$W$&$F$&$F$&$F$&$W$	
				\\
				$W$&$F$&$F$&$F$&$W$&$W$&$F$
				\\
				$W$&$W$&$F$&$W$&$W$&$W$&$W$
			\end{tabular}
		}
		\n \n \n
		\hspace*{10mm} d.h.:
		\n \n
		\hspace*{20mm}
		$(P\wedge (Q\Rightarrow P)\neq ((P\wedge Q)\vee (\neg P))$
		\n \n \hspace*{20mm}
		"Die Aussagen sind nicht gleichbedeutend."
		
		\newpage
		\section*{1.2}


					\begin{table}[h]
						\begin{tabular}{cc|cccc}
							
							$P$&$Q$&$\neg P$&$\neg Q$&$P\wedge Q$&$\neg (P\wedge Q)$ \\
							\hline
							$F$&$F$&$W$&$W$&$F$&$W$ \\
							$F$&$W$&$W$&$F$&$F$&$W$ \\
							$W$&$F$&$F$&$W$&$F$&$W$ \\
							$W$&$W$&$F$&$F$&$W$&$F$ \\
							\n \n \n
							
						\end{tabular}
						\n
						\begin{tabular}{cc|cc}
							
							$P\Rightarrow Q$&$P\vee Q$&$\neg (P\wedge \neg (Q))$&$\neg (\neg (P)\wedge \neg (Q))$ \\
							\hline
							$W$&$F$&$W$&$F$ \\
							$W$&$W$&$W$&$W$ \\
							$F$&$W$&$F$&$W$ \\
							$W$&$W$&$W$&$W$ \\
							
						\end{tabular}
						\n \n \n
						\begin{tabular}{c|ccccc}
							
							$P\dot \vee Q$&$P\Rightarrow Q$&$Q\Rightarrow P$&$\neg (P\wedge \neg (Q))$&$\neg (Q\wedge \neg (P))$&$\neg( \neg( P\wedge \neg (Q)) \wedge \neg (Q\wedge \neg (P)))$ \\\hline
							$F$&$W$&$W$&$W$&$W$&$F$ \\
							$W$&$W$&$F$&$W$&$F$&$W$ \\
							$W$&$F$&$W$&$F$&$W$&$W$ \\
							$F$&$W$&$W$&$W$&$W$&$F$ \\
							
						\end{tabular}
						\n \n \n
						\hspace*{10mm} d.h. folgende Aussagen sind Gleichbedeutend:
						\n \n
						\hspace*{20mm}
						$P\Rightarrow Q\Leftrightarrow \neg (P\wedge \neg (Q))$
						\n \n \hspace*{20mm}
						$P\vee Q\Leftrightarrow \neg (\neg (P)\wedge \neg (Q))$
						\n \n \hspace*{20mm}
						$P\dot \vee Q \Leftrightarrow \neg( \neg( P\wedge \neg (Q)) \wedge \neg (Q\wedge \neg (P)))$
					\end{table}


		\newpage
		\section*{1.3}
		\begin{table}[h]
			\begin{tabular}{cc|ccc}
				$P$&$Q$&$\neg P$&$P\wedge Q$&$P*P$ \\\hline
				$F$&$F$&$W$&$F$&$W$ \\
				$F$&$W$&$W$&$F$&$W$ \\
				$W$&$F$&$F$&$F$&$F$ \\
				$W$&$W$&$F$&$W$&$F$ \\
			\end{tabular}
			\n \n \n
			\begin{tabular}{cc|cccc}
				$P$&$Q$&$P\wedge Q$&$Q*Q$&$P*Q$&$(P*P)*(Q*Q)$ \\\hline
				$F$&$F$&$F$&$W$&$W$&$F$ \\
				$F$&$W$&$F$&$F$&$F$&$F$ \\
				$W$&$F$&$F$&$W$&$F$&$F$ \\
				$W$&$W$&$W$&$F$&$F$&$W$ \\
			\end{tabular}
			\\ \n \n \n \hspace*{10mm}
			Der Operateor "$*$" ist so zu verstehen:
			\n \n \hspace*{20mm}
			$F*F = 1,\, W*F = 0$ \n \n \hspace*{20mm}
			$W*W = 0,\, F*W = 0$
		\end{table}
\end{document}