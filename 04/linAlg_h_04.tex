\documentclass[titlepage]{article}
\usepackage{babel}
\usepackage{amsmath}
\usepackage{amssymb}
\usepackage{amsthm}
\usepackage{multicol} %spalten in seite
\usepackage{graphicx} %bilder einfügen
\usepackage{tabto} %tabulator mit \tab
\usepackage{hyperref}
\usepackage[T1]{fontenc}
\usepackage{tikz}
\usepackage{mathrsfs}  
\usepackage[utf8]{inputenc}
\usepackage{listings} %quellcode
\pagestyle{plain}
\pagenumbering{arabic}
\renewcommand{\arraystretch}{1.3} %vertikaler abstand von tabellen
\newcommand{\n}{\newline}
\usepackage[left=20mm, right=15mm, top=25mm, bottom=30mm, paper=a4paper]{geometry}

\begin{document}
	
	\title{Lineare Algebra - Übung 04}
	\author{Felix Tischler, Martrikelnummer: 191498}
	\date{\today}
	\maketitle
	
	\section*{4.1) Quadratische Gleichungen}
	$x^2+px+q=0$, lässt sich mit der pq-Formel: $x_{1,2}=-\frac{p}{2}\pm \sqrt{(\frac{p}{2})^2-q}$ lösen. Dies gilt auch für $x^2+(2\text{i}-4)\cdot x-4\text{i}=0$. mit $p=(2\text{i}-4)$ und $q=-4\text{i}$ folgt:
		\begin{align*}
			x_1&=-\frac{(2\text{i}-4)}{2}+\sqrt{\left(\frac{(2\text{i}-4)}{2}\right)^2+4\text{i}}\\
			&=2-\text{i}+\sqrt{3-4\text{i}+\text{i}}\\
			&=2+\sqrt{3}-\text{i}\\
			x_2&=-\frac{(2\text{i}-4)}{2}-\sqrt{\left(\frac{(2\text{i}-4)}{2}\right)^2+4\text{i}}\\
			&=2-\text{i}-\sqrt{3-4\text{i}+\text{i}}\\
			&=2-\sqrt{3}-\text{i}
		\end{align*}
	\underline{\underline{	D.h.: $x\in\{2+\sqrt{3}-\mathrm{i},2-\sqrt{3}-\mathrm{i}\}$}}
	\section*{4.2) Wurzelziehen in $\mathbb{C}$}
		Wenn $w\in\mathbb{C}$ und $n\in\mathbb{N}^*$, dann gilt für $z^n=w$ wie bereits in der Vorlesung definiert:
		\begin{align*}
			a)\quad z=\sqrt[n]{|w|}\left(cos(\frac{arg(w)}{n}+\varphi)+sin(\frac{arg(w)}{n}+\varphi)\mathrm{i} \right)\\
		\end{align*}
		Aus $z^5=8(\sqrt{6}-\sqrt{2})+8(\sqrt{6}+\sqrt{2})\mathrm{i}$ folgt $n=5$, $w=8(\sqrt{6}-\sqrt{2})+8(\sqrt{6}+\sqrt{2})\mathrm{i}$. Somit folgt:
		\begin{align*}
			\varphi&=\frac{2k\pi}{n},\, mit\,k\in\{0,1,2,3,4\}\\
			|w|&=\sqrt{8(\sqrt{6}-\sqrt{2})^2+8(\sqrt{6}+\sqrt{2})^2}=32\\
			arg(w)&=arccos\left(\frac{a}{\sqrt{a^2+b^2}}\right),\,mit\,a=8(\sqrt{6}-\sqrt{2}),b=8(\sqrt{6}+\sqrt{2})\\
			&=arccos\left(\frac{a}{|w|}\right)=\frac{5}{12}\pi
		\end{align*}
	Durch einsetzen in $a$) und $k=0\Rightarrow\varphi=0$ folgt:
	\begin{align*}
		Z_1:&=\sqrt[5]{32}\left(cos(\frac{\frac{5}{12}\pi}{5})+sin(\frac{\frac{5}{12}\pi}{5})\mathrm{i}\right)\\
		&=\sqrt[5]{32}\left(cos(\frac{\pi}{12})+sin(\frac{\pi}{12})\mathrm{i}\right) = 2\left(\frac{\sqrt{6}+\sqrt{2}}{4}+\frac{\sqrt{6}-\sqrt{2}}{4}\mathrm{i}\right)\\
		&=\frac{\sqrt{6}+\sqrt{2}}{2}+\frac{\sqrt{6}-\sqrt{2}}{2}\mathrm{i}
	\end{align*}

	für $k=1\Rightarrow\varphi=\frac{2\pi}{5}$ folgt:
	\begin{align*}
		Z_2&=2\left(cos(\frac{\frac{5}{12}\pi}{5}+\frac{2\pi}{5})+sin(\frac{\frac{5}{12}\pi}{5}+\frac{2\pi}{5})\mathrm{i}\right)\\
		&=2(cos(\frac{29}{60}\pi)+sin(\frac{29}{60}\pi)\mathrm{i})
	\end{align*}

	für $k=2\Rightarrow\varphi=\frac{4\pi}{5}$ folgt:
	\begin{align*}
		Z_3&=2\left(cos(\frac{\frac{5}{12}\pi}{5}+\frac{4\pi}{5})+sin(\frac{\frac{5}{12}\pi}{5}+\frac{4\pi}{5})\mathrm{i}\right)\\
		&=2(cos(\frac{53}{60}\pi)+sin(\frac{53}{60}\pi)\mathrm{i})
	\end{align*}

	für $k=3\Rightarrow\varphi=\frac{6\pi}{5}$ folgt:
	\begin{align*}
		Z_4&=2\left(cos(\frac{\frac{5}{12}\pi}{5}+\frac{6\pi}{5})+sin(\frac{\frac{5}{12}\pi}{5}\frac{6\pi}{5})\mathrm{i}\right)\\
		&=2(cos(\frac{77}{60}\pi)+sin(\frac{77}{60}\pi)\mathrm{i})
	\end{align*}

	für $k=4\Rightarrow\varphi=\frac{8\pi}{5}$ folgt:
	\begin{align*}
		Z_5&=2\left(cos(\frac{\frac{5}{12}\pi}{5}+\frac{8\pi}{5})+sin(\frac{\frac{5}{12}\pi}{5}+\frac{8\pi}{5})\mathrm{i}\right)\\
		&=2(cos(\frac{101}{60}\pi)+sin(\frac{101}{60}\pi)\mathrm{i})
	\end{align*}

	\section*{4.3) Matrixprodukte}
		Sei $A:=\left(\begin{array}{ccc}
			1&-1&2\\
			0&3&5\\
			1&8&-7
		\end{array}\right)$, 
		$B:=\left(\begin{array}{cccc}
			-1&0&1&0\\
			0&1&0&-1\\
			1&0&-1&0
		\end{array}\right)$,
		$C:=\left(\begin{array}{c}
			1\\
			0\\
			8\\
			-7
		\end{array}\right)$,
		$D:=\left(\begin{array}{cccc}
			-1&2&0&8
		\end{array}\right)$, \\dann sind alle möglichen Produkte:\footnote{die dazugehörigen Rechnungen finden sie im Anhang in Moodle.}\\
	
		$AA=\left(\begin{array}{ccc}
			1&-1&2\\
			0&3&5\\
			1&8&-7
			\end{array}\right)
			\left(\begin{array}{ccc}
			1&-1&2\\
			0&3&5\\
			1&8&-7
			\end{array}\right)=
			\left(\begin{array}{ccc}
				3&12&-17\\
				5&49&-20\\
				-6&-33&91
			\end{array}\right)$\\\\
		
		$AB=\left(\begin{array}{ccc}
			1&-1&2\\
			0&3&5\\
			1&8&-7
		\end{array}\right)
		\left(\begin{array}{cccc}
			-1&0&1&0\\
			0&1&0&-1\\
			1&0&-1&0
		\end{array}\right)=
		\left(\begin{array}{cccc}
			1&-1&-1&1\\
			5&3&-5&-3\\
			-8&8&8&-8
		\end{array}\right)$\\
	
		$BC=\left(\begin{array}{cccc}
			-1&0&1&0\\
			0&1&0&-1\\
			1&0&-1&0
		\end{array}\right)
		\left(\begin{array}{c}
			1\\
			0\\
			8\\
			-7
		\end{array}\right)=
		\left(\begin{array}{c}
			7\\
			7\\
			-7
		\end{array}\right)$$\\$\\
	
		$CD=\left(\begin{array}{ccc}
			1\\
			0\\
			8\\
			-7
		\end{array}\right)
		\left(\begin{array}{cccc}
			-1&2&0&8
		\end{array}\right)=
		\left(\begin{array}{cccc}
			-1&2&0&8\\
			0&0&0&0\\
			-8&16&0&64\\
			7&-14&0&-56
		\end{array}\right)$$\\$\\
	
		$DC=\left(\begin{array}{cccc}
			-1&2&0&8
		\end{array}\right)
		\left(\begin{array}{ccc}
			1\\
			0\\
			8\\
			-7
		\end{array}\right)=
		\left(\begin{array}{c}
			-57
		\end{array}\right)$$\\$
		
	\newpage
	\section*{4.4) Matrixarithmetik}
	Es sei $A\in\mathbb{K}^{m\times n},B\in\mathbb{K}^{n\times p},C\in\mathbb{K}^{p\times q}$. Des weiteren sind $i,l,k,t\in\mathbb{N}^*$ Zunächst betrachtet man die Dimensionen:
	\begin{align*}
		AB\in\mathbb{K}^{m\times p} &\Rightarrow(AB)C\in\mathbb{K}^{m\times q}\\
		BC\in\mathbb{K}^{n\times q} &\Rightarrow A(BC)\in\mathbb{K}^{m\times q}
	\end{align*}
	Die Dimensionen stimmen überein, somit gilt für alle $i\le m$ und $l\le q$:
	\begin{align*}
		((AB)C)_{il}&=(A(BC))_{il}\\
		\sum_{k=1}^n(AB)_{i,k}C_{k,l}&=\sum_{k=1}^pA_{i,k}(BC)_{k,l}\\
		\sum_{k=1}^n\left(\sum_{t=1}^pA_{i,t}B_{t,k}\right)C_{k,l}&=\sum_{k=1}^pA_{i,k}\left(\sum_{t=1}^nB_{k,t}C_{t,l}\right)\\
		\overset{Distr.}{\Leftrightarrow}
		\sum_{k=1}^n\left(\sum_{t=1}^pA_{i,t}B_{t,k}C_{k,l}\right)&=\sum_{k=1}^p\left(\sum_{t=1}^nA_{i,k}B_{k,t}C_{t,l}\right)\\
		\overset{Komm.}{\Rightarrow}
		&=\sum_{k=1}^n\left(\sum_{t=1}^pA_{i,t}B_{t,k}C_{k,l}\right)\\
		&\quad\quad\\
		&=((AB)C)_{i,l}\qed	
	\end{align*}
	In den Schritten bezüglich der Kommutativität und Distributivität wurde das Distributivitätsgesetz und die Kommutativität der Addition von $\mathbb{K}$ vorausgesetzt.
\end{document}
